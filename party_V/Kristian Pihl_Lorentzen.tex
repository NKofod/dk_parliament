%!TEX TS-program = xelatex
%!TEX encoding = UTF-8 Unicode
\documentclass[11pt, a4paper]{awesome-cv}
\geometry{left=1.4cm, top=.8cm, right=1.4cm, bottom=1.8cm, footskip=.5cm}
\fontdir[fonts/]
\colorlet{awesome}{V-colour}
\setbool{acvSectionColorHighlight}{true}
\renewcommand{\acvHeaderSocialSep}{\quad\textbar\quad}
\recipient{}{}
\name{Kristian Pihl}{Lorentzen}
\mobile{+45 2752 2818}
\email{kristian.lorentzen@ft.dk}
\position{Medlem af Folketinget{\enskip\cdotp\enskip}Venstre}
\address{}
\photo[circle,noedge,left]{"./Kristian Pihl_Lorentzen_profile.jpg"}
\letterdate{\today}
\lettertitle{Kristian Pihl Lorentzen - Blå Bog}
\letteropening{}
\letterclosing{}
\letterenclosure[Attached]{Stemme Statistik}
\begin{document}
\makecvheader[R]
\makecvfooter{\today}{\lettertitle{Kristian Pihl Lorentzen - Blå Bog}}{}
\makelettertitle
\begin{cvletter}
\lettersection{Baggrund}
Kristian Pihl Lorentzen, født 19. maj 1961 i Silkeborg, søn af vejformand Preben Pihl Pedersen og bogholder Anna Lorentzen. Samlevende med kørelærer Carina Bach Lauritsen. Har tre børn, Julie født i 1987, Sofie født i 1990 og Frederik født i 1993. 

\lettersection{Uddannelse}
\begin{itemize}
\item Videregående officersuddannelse, Forsvarsakademiet og Hærens Officersskole, 1989-1990.
\item Uddannelse til officer af linjen, Hærens Officersskole, 1983-1986.
\item Uddannelse til officer af reserven, 1981-1982.
\item Student, Viborg Amtsgymnasium, 1978-1981.
\item Realeksamen, Ans Skole, 1975-1978.
\end{itemize}
\lettersection{Parlamentarisk Karriere}
\subsection*{Ordførerskaber}
\begin{itemize}
\item Transportordfører
\end{itemize}
\subsection*{Parlamentariske Tillidsposter}
\begin{itemize}
\item Næstformand for Dansk Interparlamentarisk Gruppes bestyrelse 2019-2020.
\item Næstformand for Dansk Interparlamentarisk Gruppes bestyrelse 2015-2016.
\item Formand for Dansk Interparlamentarisk Gruppes bestyrelse 2007-2013.
\item Formand for den danske delegation til OSCE’s Parlamentariske Forsamling 2005-2007.
\item Næstformand for den danske delegation til Nordisk Råd 2005-2007.
\end{itemize}
\subsection*{Folketinget}
\subsubsection*{Medlemsperioder}
\begin{itemize}
\item Folketingsmedlem for Venstre i Vestjyllands Storkreds fra 13. november 2007.
\item Folketingsmedlem for Venstre i Viborg Amtskreds 8. februar 2005 - 13. november 2007.
\end{itemize}
\subsubsection*{Kandidaturer}
\begin{itemize}
\item Kandidat for Venstre i Viborg Østkredsen fra 2007.
\item Kandidat for Venstre i Viborgkredsen 1999-2006.
\end{itemize}
\subsection*{Folketingets Præsidium}
\begin{itemize}
\item Medlem af Folketingets Præsidium 3. juli 2015 - 12. august 2019.
\end{itemize}
\lettersection{Erhvervserfaring}
\begin{itemize}
\item EU-observatør, tidligere Jugoslavien, 2003-2003.
\item Selvstændig konsulentvirksomhed, 1999-2005.
\item Ansat i Forsvaret, 1981-2005.
\end{itemize}
\lettersection{Publikationer}
Har skrevet »På sporet af Danmark ‒ Jernbanen før, nu og i fremtiden«, 2013 og »Hvor der er vilje, er der vej ‒ Transportpolitiske visioner for Danmark«, 2010. Medforfatter til »Stop trafikal egoisme«, 2020. 

\end{cvletter}
\end{document}