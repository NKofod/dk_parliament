%!TEX TS-program = xelatex
%!TEX encoding = UTF-8 Unicode
\documentclass[11pt, a4paper]{awesome-cv}
\geometry{left=1.4cm, top=.8cm, right=1.4cm, bottom=1.8cm, footskip=.5cm}
\fontdir[fonts/]
\colorlet{awesome}{V-colour}
\setbool{acvSectionColorHighlight}{true}
\renewcommand{\acvHeaderSocialSep}{\quad\textbar\quad}
\recipient{}{}
\name{Preben Bang}{Henriksen}
\mobile{+45 3337 4514}
\email{bang.henriksen@ft.dk}
\position{Medlem af Folketinget{\enskip\cdotp\enskip}Venstre}
\address{}
\photo[circle,noedge,left]{"./Preben Bang_Henriksen_profile.jpg"}
\letterdate{\today}
\lettertitle{Preben Bang Henriksen - Blå Bog}
\letteropening{}
\letterclosing{}
\letterenclosure[Attached]{Stemme Statistik}
\begin{document}
\makecvheader[R]
\makecvfooter{\today}{\lettertitle{Preben Bang Henriksen - Blå Bog}}{}
\makelettertitle
\begin{cvletter}
\lettersection{Baggrund}
Preben Bang Henriksen, født 11. februar 1954, søn af snedkermester Svend Bang Henriksen og hustru Gerda Margrethe Henriksen. Gift med Anna Marie Høstgaard. Parret har børnene Svend, født i 1986, og Margrethe, født i 1987.

\lettersection{Uddannelse}
\begin{itemize}
\item Cand.jur., Aarhus Universitet, 1979-1979.
\item Student, Nørresundby Gymnasium, 1974-1974.
\item Garden City High School, New York, 1973-1973.
\end{itemize}
\lettersection{Parlamentarisk Karriere}
\subsection*{Ordførerskaber}
\begin{itemize}
\item Retsordfører
\end{itemize}
\subsection*{Parlamentariske Tillidsposter}
\begin{itemize}
\item Retsordfører fra 2021.
\item Medlem af bestyrelsen for Venstres Folketingsgruppe fra 2020.
\item Næstformand for Venstres folketingsgruppe 2020-2021.
\item Formand for Retsudvalget 2019-2021.
\item Næstformand for Udvalget vedrørende Efterretningstjenesterne 2015-2019.
\item Retsordfører 2015-2019.
\end{itemize}
\subsection*{Folketinget}
\subsubsection*{Medlemsperioder}
\begin{itemize}
\item Folketingsmedlem for Venstre i Nordjyllands Storkreds fra 15. september 2011.
\end{itemize}
\subsubsection*{Kandidaturer}
\begin{itemize}
\item Kandidat for Venstre i Aalborg Nordkredsen fra 2011.
\end{itemize}
\lettersection{Erhvervserfaring}
\begin{itemize}
\item Advokat (H), advokatfirmaet Børge Nielsen, fra 2011.
\item Møderet for Højesteret, fra 1986.
\item Selvstændig advokatvirksomhed, fra 1983.
\item Møderet for Landsretten, fra 1982.
\item Advokat, fra 1982.
\item Advokatfuldmægtig, landsretssagfører Bent Kinnerup, 1979-1982.
\end{itemize}
\lettersection{Publikationer}
Har skrevet »Erhvervslejeretten i hovedtræk«, 1990, »Erhvervslejeretten«, 1997, »Skolens lejemål«, 1999, og »Erhvervslejemål - rettigheder og pligter efter den ny erhvervslejelov«, 2008. Har skrevet diverse artikler om ejendomshandel og lejeret i Ugeskrift for Retsvæsen og Tidsskrift for Bolig- og Byggeret.

\end{cvletter}
\end{document}