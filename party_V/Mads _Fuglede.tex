%!TEX TS-program = xelatex
%!TEX encoding = UTF-8 Unicode
\documentclass[11pt, a4paper]{awesome-cv}
\geometry{left=1.4cm, top=.8cm, right=1.4cm, bottom=1.8cm, footskip=.5cm}
\fontdir[fonts/]
\colorlet{awesome}{V-colour}
\setbool{acvSectionColorHighlight}{true}
\renewcommand{\acvHeaderSocialSep}{\quad\textbar\quad}
\recipient{}{}
\name{Mads }{Fuglede}
\mobile{+45 6162 5274}
\email{mads.fuglede@ft.dk}
\position{Medlem af Folketinget{\enskip\cdotp\enskip}Venstre}
\address{}
\photo[circle,noedge,left]{"./party_V/Mads _Fuglede_profile.jpg"}
\letterdate{\today}
\lettertitle{Mads  Fuglede - Blå Bog}
\letteropening{}
\letterclosing{}
\letterenclosure[Attached]{Stemme Statistik}
\begin{document}
\makecvheader[R]
\makecvfooter{\today}{\lettertitle{Mads  Fuglede - Blå Bog}}{}
\makelettertitle
\begin{cvletter}
\lettersection{Baggrund}
Mads Fuglede, født 23. august 1971 på Bispebjerg Hospital, søn af læge Niels Christian Fuglede og oversætter (ED og Ba i fransk og engelsk) Dorthe Fuglede. Samlevende med Maria Worm. Parret har sønnen Johannes.

\lettersection{Uddannelse}
\begin{itemize}
\item Cand.mag. i historie og filosofi, Aarhus Universitet, 2004-2004.
\item Student, Dronninglund Gymnasium, 1989-1989.
\end{itemize}
\lettersection{Parlamentarisk Karriere}
\subsection*{Ordførerskaber}
\begin{itemize}
\item Integrationsordfører
\item Udlændingeordfører
\end{itemize}
\subsection*{Folketinget}
\subsubsection*{Medlemsperioder}
\begin{itemize}
\item Folketingsmedlem for Venstre i Københavns Omegns Storkreds fra 5. juni 2019.
\end{itemize}
\subsubsection*{Kandidaturer}
\begin{itemize}
\item Kandidat for Venstre i Gladsaxekredsen fra 2014.
\end{itemize}
\lettersection{Erhvervserfaring}
\begin{itemize}
\item Fast kommentator indenfor amerikansk politik og ved de amerikanske præsidentvalg i 2008 og 2012 og medvært for det amerikanske præsidentvalg i 2016, TV 2, fra 2006.
\item Underviser i international politik og amerikansk politik, ved diverse højskoler og universiteter, 1998-2014.
\item Forfatter og foredragsholder,.
\end{itemize}
\lettersection{Publikationer}
Har skrevet »USA - den universelle nation«, Gyldendal, 2008.

\end{cvletter}
\end{document}