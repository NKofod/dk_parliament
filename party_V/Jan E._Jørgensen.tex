%!TEX TS-program = xelatex
%!TEX encoding = UTF-8 Unicode
\documentclass[11pt, a4paper]{awesome-cv}
\geometry{left=1.4cm, top=.8cm, right=1.4cm, bottom=1.8cm, footskip=.5cm}
\fontdir[fonts/]
\colorlet{awesome}{V-colour}
\setbool{acvSectionColorHighlight}{true}
\renewcommand{\acvHeaderSocialSep}{\quad\textbar\quad}
\recipient{}{}
\name{Jan E.}{Jørgensen}
\mobile{+45 3337 4551}
\email{jan.e@ft.dk}
\position{Medlem af Folketinget{\enskip\cdotp\enskip}Venstre}
\address{}
\photo[circle,noedge,left]{"./party_Venstre/Jan E._Jørgensen_profile.jpg"}
\letterdate{\today}
\lettertitle{Jan E. Jørgensen - Blå Bog}
\letteropening{}
\letterclosing{}
\letterenclosure[Attached]{Stemme Statistik}
\begin{document}
\makecvheader[R]
\makecvfooter{\today}{\lettertitle{Jan E. Jørgensen - Blå Bog}}{}
\makelettertitle
\begin{cvletter}
\lettersection{Baggrund}
Jan Ejnar Jørgensen, født 19. februar 1965 på Frederiksberg, søn af Einar Jacob Sørensen og Asta Noomi Jørgensen. Gift med Jane Nørgaard Jørgensen. Far til Kamma Nørgaard Jørgensen og Valdemar Nørgaard Jørgensen.

\lettersection{Uddannelse}
\begin{itemize}
\item Juridisk embedseksamen, Københavns Universitet, 1989-1995.
\item Samfundssproglig studentereksamen, Sankt Annæ Gymnasium, 1981-1984.
\end{itemize}
\lettersection{Parlamentarisk Karriere}
\subsection*{Ordførerskaber}
\begin{itemize}
\item EU-ordfører
\end{itemize}
\subsection*{Parlamentariske Tillidsposter}
\begin{itemize}
\item Næstformand for Indfødsretsudvalget fra 2019.
\item EU-ordfører fra 2016.
\item Tidligere offentlighedsordfører, menneskerettighedsordfører, indfødsretsordfører, socialordfører og kommunalordfører.
\end{itemize}
\subsection*{Folketinget}
\subsubsection*{Medlemsperioder}
\begin{itemize}
\item Folketingsmedlem for Venstre i Københavns Storkreds fra 15. september 2011.
\end{itemize}
\subsubsection*{Kandidaturer}
\begin{itemize}
\item Kandidat for Venstre i Slotskredsen fra 2011.
\item Kandidat for Venstre i Utterslevkredsen 2006-2011.
\item Kandidat for Venstre i Østerbrokredsen 1993-1995.
\item Kandidat for Venstre i Brønshøjkredsen 1993-1995.
\end{itemize}
\lettersection{Erhvervserfaring}
\begin{itemize}
\item Advokat (L), DLA Nordic og Horten, 2005-2011.
\item Miljøkonsulent, advokatfuldmægtig, advokat (L), De Samvirkende Købmænd, 1997-2005.
\item Direktør, Foreningen for et Bedre Butiksmiljø, 1995-1997.
\item Annoncekonsulent, Vibenhus Butikscenters avis, 1989-1995.
\item Annoncekonsulent , Farum Nyt, 1985-1988.
\item Journalist, Frederiksberg Posten, 1985-1985.
\item Sekretær, Venstres Ungdom, 1984-1986.
\item Folkeskolevikar, Rådmandsgades og Stevnsgades Skoler, 1984-1985.
\item Abonnementsinspektør, Berlingske Tidende, 1984-1985.
\item Kirkesanger, Korsvejs Kirke, Johannes Døbers Kirke, Høje Gladsaxe kirke, 1981-1986.
\end{itemize}
\lettersection{Publikationer}
Medforfatter til »En ægte liberal &ndash; venstremanden, der ikke vil tie stille«, 2018. Blogger på Jyllands-Posten fra 2018. Blogger på Berlingske, Den politiske puls, 2016-2018.

\end{cvletter}
\end{document}