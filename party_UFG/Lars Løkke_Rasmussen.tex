%!TEX TS-program = xelatex
%!TEX encoding = UTF-8 Unicode
\documentclass[11pt, a4paper]{awesome-cv}
\geometry{left=1.4cm, top=.8cm, right=1.4cm, bottom=1.8cm, footskip=.5cm}
\fontdir[fonts/]
\colorlet{awesome}{UFG-colour}
\setbool{acvSectionColorHighlight}{true}
\renewcommand{\acvHeaderSocialSep}{\quad\textbar\quad}
\recipient{}{}
\name{Lars Løkke}{Rasmussen}
\mobile{}
\email{lars.loekke@ft.dk}
\position{Fhv. statsminister{\enskip\cdotp\enskip}Uden for folketingsgrupperne}
\address{}
\photo[circle,noedge,left]{"./party_UFG/Lars Løkke_Rasmussen_profile.jpg"}
\letterdate{\today}
\lettertitle{Lars Løkke Rasmussen - Blå Bog}
\letteropening{}
\letterclosing{}
\letterenclosure[Attached]{Stemme Statistik}
\begin{document}
\makecvheader[R]
\makecvfooter{\today}{\lettertitle{Lars Løkke Rasmussen - Blå Bog}}{}
\makelettertitle
\begin{cvletter}
\lettersection{Baggrund}
Lars Løkke Rasmussen, født 15. maj 1964 i Vejle, søn af regnskabschef Jeppe Løkke Rasmussen og husmoder Lise Løkke Rasmussen.

\lettersection{Uddannelse}
\begin{itemize}
\item Cand.jur., Københavns Universitet, 1992-1992.
\item Samfundsfaglig-matematisk student, Helsinge Gymnasium, 1983-1983.
\item Folkeskolens afgangsprøve, Græsted Skole, 1980-1980.
\end{itemize}
\lettersection{Parlamentarisk Karriere}
\subsection*{Ministerposter}
\begin{itemize}
\item Statsminister 28. juni 2015 - 27. juni 2019.
\item Statsminister 5. april 2009 - 3. oktober 2011.
\item Finansminister 23. november 2007 - 7. april 2009.
\item Indenrigs- og sundhedsminister 27. november 2001 - 23. november 2007.
\end{itemize}
\subsection*{Parlamentariske Tillidsposter}
\begin{itemize}
\item Formand for Venstre 2009-2019.
\item Næstformand for Venstre 1998-2009.
\end{itemize}
\subsection*{Folketinget}
\subsubsection*{Medlemsperioder}
\begin{itemize}
\item Folketingsmedlem for Uden for folketingsgrupperne i Sjællands Storkreds fra 1. januar 2021.
\item Folketingsmedlem for Venstre i Sjællands Storkreds 18. juni 2015 - 31. december 2020.
\item Folketingsmedlem for Venstre i Nordsjællands Storkreds 13. november 2007 - 18. juni 2015.
\item Folketingsmedlem for Venstre i Frederiksborg Amtskreds 21. september 1994 - 13. november 2007.
\end{itemize}
\subsubsection*{Kandidaturer}
\begin{itemize}
\item Kandidat for Venstre i Køgekredsen 2012-2020.
\item Kandidat for Venstre i Frederikssundkredsen 2007-2012.
\item Kandidat for Venstre i Frederiksværkkredsen 1986-2006.
\end{itemize}
\lettersection{Erhvervserfaring}
\begin{itemize}
\item Selvstændig konsulentvirksomhed, 1990-1995.
\end{itemize}
\lettersection{Publikationer}
Har skrevet »Hvis jeg bli&rsquo;&acute;r gammel ‒ en ældrepolitisk debatbog«, 1997. Medforfatter til »Foreningshåndbogen«, 1994, revideret 2006. Medvirkende i  »Befrielsens øjeblik ‒ samtaler med Lars Løkke Rasmussen« af Kirsten Jacobsen, 2019, »Obama ‒ håbets præsident«, 2019. »I orkanens øje - samtaler med statsminister Lars Løkke Rasmussen« af Thomas Larsen, 2010 og »Løkkeland - Lars Løkke Rasmussens Danmark« af Kirsten Jacobsen, 2006.

\end{cvletter}
\end{document}