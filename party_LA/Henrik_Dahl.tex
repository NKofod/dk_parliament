%!TEX TS-program = xelatex
%!TEX encoding = UTF-8 Unicode
\documentclass[11pt, a4paper]{awesome-cv}
\geometry{left=1.4cm, top=.8cm, right=1.4cm, bottom=1.8cm, footskip=.5cm}
\fontdir[fonts/]
\colorlet{awesome}{LA-colour}
\setbool{acvSectionColorHighlight}{true}
\renewcommand{\acvHeaderSocialSep}{\quad\textbar\quad}
\recipient{}{}
\name{Henrik}{Dahl}
\mobile{+45 3337 4912}
\email{henrik.dahl@ft.dk}
\position{Medlem af Folketinget{\enskip\cdotp\enskip}Liberal Alliance}
\address{}
\photo[circle,noedge,left]{"./party_Liberal Alliance/Henrik_Dahl_profile.jpg"}
\letterdate{\today}
\lettertitle{Henrik Dahl - Blå Bog}
\letteropening{}
\letterclosing{}
\letterenclosure[Attached]{Stemme Statistik}
\begin{document}
\makecvheader[R]
\makecvfooter{\today}{\lettertitle{Henrik Dahl - Blå Bog}}{}
\makelettertitle
\begin{cvletter}
\lettersection{Baggrund}
Johan Henrik Dahl, født 20. februar 1960 i Vejlby-Risskov, søn af sognepræst Paul Dahl og lektor Inger Marie Dahl. Samlevende med Christina Yoon Petersen.

\lettersection{Uddannelse}
\begin{itemize}
\item Ph.d., Copenhagen Business School, 1990-1993.
\item Cand.scient.soc. (sociolog), Københavns Universitet, 1980-1987.
\item Sprogofficer (russisk), København, 1978-1980.
\item MA in Communications, University of Pennsylvania, Philadelphia, USA, fra 1985 til 1986 og fra 1987 til 1988,.
\end{itemize}
\lettersection{Parlamentarisk Karriere}
\subsection*{Ordførerskaber}
\begin{itemize}
\item Ældreordfører
\item Børneordfører
\item Fiskeriordfører
\item Fødevareordfører
\item Forskningsordfører
\item Forsvarsordfører
\item Indfødsretsordfører
\item Integrationsordfører
\item Kirkeordfører
\item Kulturordfører
\item Miljøordfører
\item Sundhedsordfører
\item Udenrigsordfører
\item Udlændingeordfører
\item Undervisningsordfører
\item Uddannelsesordfører
\end{itemize}
\subsection*{Parlamentariske Tillidsposter}
\begin{itemize}
\item Næstformand for Uddannelses- og Forskningsudvalget 2015-2019.
\end{itemize}
\subsection*{Folketinget}
\subsubsection*{Medlemsperioder}
\begin{itemize}
\item Folketingsmedlem for Liberal Alliance i Sydjyllands Storkreds fra 18. juni 2015.
\end{itemize}
\subsubsection*{Kandidaturer}
\begin{itemize}
\item Kandidat for Liberal Alliance i alle opstillingskredse i Sydjyllands Storkreds fra 2014.
\end{itemize}
\lettersection{Erhvervserfaring}
\begin{itemize}
\item Forfatter og foredragsholder, København, 2009-2015.
\item Anmelder, Weekendavisen, 2009-2014.
\item Udviklingschef, Niras A/S, Allerød, 2008-2009.
\item Studievært, DR P1, København, 2004-2005.
\item Adjungeret professor, Copenhagen Business School, 2003-2008.
\item Medindehaver, Explora A/S, København, 1999-2008.
\item Forskningschef, AC Nielsen AIM, København, 1994-1998.
\item Adjunkt, Roskilde Universitetscenter, 1993-1994.
\item Adjunktvikar, Copenhagen Business School, 1990-1993.
\item Medieforsker, Danmarks Radio, 1988-1990.
\end{itemize}
\lettersection{Publikationer}
Forfatter til romanen »NT«, People&rsquo;s Press, 2013, »Spildte kræfter«, Gyldendal, 2011, »Den usynlige verden«, Gyldendal, 2008, »Mindernes land«, Gyldendal, 2005, »Den kronologiske uskyld«, Gyldendal, 1998, og »Hvis din nabo var en bil«, Akademisk Forlag, 1997. Medforfatter til »Sandheden kort - Christiansborg fra A til Å«, People's Press, 2018, »Krigeren, borgeren og taberen«, Gyldendal, 2006, »Epostler«, Gyldendal, 2003, »Det ny systemskifte«, Gyldendal, 2001, »Borgerlige ord efter revolutionen«, Gyldendal, 1999, og »Marketing og semiotik«, Akademisk Forlag, 1993. Tillige forfatter til en lang række bidrag til antologier og dagblade fra 1995.

\end{cvletter}
\end{document}