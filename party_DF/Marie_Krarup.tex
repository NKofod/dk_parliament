%!TEX TS-program = xelatex
%!TEX encoding = UTF-8 Unicode
\documentclass[11pt, a4paper]{awesome-cv}
\geometry{left=1.4cm, top=.8cm, right=1.4cm, bottom=1.8cm, footskip=.5cm}
\fontdir[fonts/]
\colorlet{awesome}{DF-colour}
\setbool{acvSectionColorHighlight}{true}
\renewcommand{\acvHeaderSocialSep}{\quad\textbar\quad}
\recipient{}{}
\name{Marie}{Krarup}
\mobile{+45 3337 5134}
\email{marie.krarup@ft.dk}
\position{Medlem af Folketinget{\enskip\cdotp\enskip}Dansk Folkeparti}
\address{}
\photo[circle,noedge,left]{"./Marie_Krarup_profile.jpg"}
\letterdate{\today}
\lettertitle{Marie Krarup - Blå Bog}
\letteropening{}
\letterclosing{}
\letterenclosure[Attached]{Stemme Statistik}
\begin{document}
\makecvheader[R]
\makecvfooter{\today}{\lettertitle{Marie Krarup - Blå Bog}}{}
\makelettertitle
\begin{cvletter}
\lettersection{Baggrund}
Marie Krarup Soelberg, født 6. december 1965 i Seem ved Ribe, datter af sognepræst Søren Krarup og husmoder Elisabeth Krarup. Har to børn.

\lettersection{Uddannelse}
\begin{itemize}
\item Pædagogikum, Odsherred Gymnasium, 2006-2007.
\item Sidefag i religion, Åbent Universitet, Københavns Universitet, 2006-2006.
\item Hd 1. del og fag på hd 2. del, Copenhagen Business School, 2003-2003.
\item Cand.mag. i øststatskundskab og samfundsfag, Københavns Universitet og Aarhus Universitet, 1996-1996.
\item Reserveofficer, grunduddannelse (russisk), Svanemøllens Kaserne, 1985-1987.
\item Første år på teologistudiet, Aarhus Universitet, 1984-1985.
\item Klassisksproglig student, Ribe Katedralskole, 1984-1984.
\end{itemize}
\lettersection{Parlamentarisk Karriere}
\subsection*{Ordførerskaber}
\begin{itemize}
\item Ordfører vedr. det danske mindretal
\item Børneordfører
\item Gymnasieordfører
\item Indfødsretsordfører
\item Integrationsordfører
\item Ordfører vedr. det tyske mindretal
\item Undervisningsordfører
\end{itemize}
\subsection*{Parlamentariske Tillidsposter}
\begin{itemize}
\item Formand for Indfødsretsudvalget fra 2019.
\item Børne- og undervisningsordfører fra 2019.
\item Indfødsretsordfører fra 2019.
\item Kirkeordfører fra 2019.
\item Ordfører for det danske mindretal (Sydslesvig) og ordfører for det tyske mindretal fra 2019.
\item Integrationsordfører fra 2018.
\item Gymnasieordfører fra 2012.
\item Beredskabsordfører 2011-2015.
\item Forsvarsordfører 2011-2018.
\end{itemize}
\subsection*{Folketinget}
\subsubsection*{Medlemsperioder}
\begin{itemize}
\item Folketingsmedlem for Dansk Folkeparti i Sydjyllands Storkreds fra 15. september 2011.
\end{itemize}
\subsubsection*{Kandidaturer}
\begin{itemize}
\item Kandidat for Dansk Folkeparti i Vardekredsen fra 2013.
\item Kandidat for Dansk Folkeparti i Esbjerg Omegnskredsen 2010-2013.
\end{itemize}
\lettersection{Erhvervserfaring}
\begin{itemize}
\item Forelæser i emner om Rusland, Folkeuniversitetet, fra 2019.
\item Rejseleder i Rusland, Akademisk Rejsebureau, fra 2019.
\item Lektor, Frederiksborg Gymnasium, 2007-2011.
\item Analytiker, ekspert i Østeuropa og SNG, Eksportkreditfonden, 2003-2005.
\item Markedsanalytiker, House of Prince, 2001-2002.
\item Assisterende forsvarsattache, major, den danske ambassade i Moskva, 1998-2001.
\item Fuldmægtig, Forsvarets Internationale Afdeling, 1997-1998.
\item Underviser, VUC Falster, Greve Gymnasium og Hærens Specialskole, 1996-1997.
\item Reserveofficer, 1987-1997.
\end{itemize}
\lettersection{Publikationer}
Har skrevet »Ny kold krig - Marie Krarup taler med 17 eksperter fra øst til vest«, 2018. Redaktør sammen med Monica Papazu af »Den politiske korrekthed og virkelighedens forsvinden - filosofiske og politiske analyser i lyst af den nye kolde krig mod Rusland«, 2019.

\end{cvletter}
\end{document}