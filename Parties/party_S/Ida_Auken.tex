%!TEX TS-program = xelatex
%!TEX encoding = UTF-8 Unicode
\documentclass[11pt, a4paper]{awesome-cv}
\geometry{left=1.4cm, top=.8cm, right=1.4cm, bottom=1.8cm, footskip=.5cm}
\fontdir[fonts/]
\colorlet{awesome}{S-colour}
\setbool{acvSectionColorHighlight}{true}
\renewcommand{\acvHeaderSocialSep}{\quad\textbar\quad}
\recipient{}{}
\name{Ida}{Auken}
\mobile{+45 3337 4040}
\email{ida.aukenft.dk}
\position{Fhv. minister{\enskip\cdotp\enskip}Socialdemokratiet}
\address{}
\photo[circle,noedge,left]{"./party_S/Ida_Auken_profile.jpg"}
\letterdate{\today}
\lettertitle{Ida Auken - Blå Bog}
\letteropening{}
\letterclosing{}
\letterenclosure[Attached]{Stemme Statistik}
\begin{document}
\makecvheader[R]
\makecvfooter{\today}{\lettertitle{Ida Auken - Blå Bog}}{}
\makelettertitle
\begin{cvletter}
\lettersection{Baggrund}
Ida Margrete Meier Auken, foslashdt 22. april 1978 paring Frederiksberg, datter af professor Erik A. Nielsen og medlem af EuropaParlamentet Margrete Auken.nbspGift med professor Bent Meier Soslashrensen. Mor til Niels og Simon Meier Auken.

\lettersection{Uddannelse}
\begin{itemize}
\item Cand.theol., Københavns Universitet,19982006.
\end{itemize}
\lettersection{Parlamentarisk Karriere}
\subsection*{Ministerposter}
\begin{itemize}
\item Miljøminister3. oktober 2011  3. februar 2014.
\end{itemize}
\subsection*{Ordførerskaber}
\begin{itemize}
\item Forskningsordfører
\item Hovedstadsordfører
\item Kulturordfører
\end{itemize}
\subsection*{Parlamentariske Tillidsposter}
\begin{itemize}
\item Formand for Klima, Energi og Forsyningsudvalget20192020.
\item Formand for Miljø og Planlægningsudvalget20092011.
\item Tidligere uddannelsesordfører, klima, energi, og forsyningsordfører, miljøordfører, erhvervsordfører, iværksætteriordfører, forbrugerordfører og landbrugsordfører.
\end{itemize}
\subsection*{Folketinget}
\subsubsection*{Medlemsperioder}
\begin{itemize}
\item Folketingsmedlem for Socialdemokratiet i Københavns Storkreds fra 3. februar 2021.
\item Folketingsmedlem for Socialdemokratiet i Københavns Storkredsfra 3. februar 2021.
\item Folketingsmedlem for Radikale Venstre i Københavns Storkreds4. februar 2014  2. februar 2021.
\item Folketingsmedlem for Socialistisk Folkeparti i Københavns Storkreds13. november 2007  3. februar 2014.
\end{itemize}
\subsubsection*{Kandidaturer}
\begin{itemize}
\item Kandidat for Socialdemokratiet i Slotskredsenfra 2021.
\item Kandidat for Radikale Venstre i Vesterbrokredsen20142021.
\item Kandidat for Socialistisk Folkeparti i Valbykredsen20072014.
\item Kandidat for Socialistisk Folkeparti i Vesterbrokredsen20072014.
\item Kandidat for Socialistisk Folkeparti i Tårnbykredsenfra 2007.
\item Kandidat for Socialistisk Folkeparti i Esbjergkredsen20042006.
\end{itemize}
\lettersection{Erhvervserfaring}
\begin{itemize}
\item Ekstern lektor, Det Teologiske Fakultet, Københavns Universitet,20062007.
\item Forlagsredaktør, Forlaget Alfa,20042007.
\end{itemize}
\lettersection{Publikationer}
Forfatter til raquoTro paring det  klima, haringb og handlekraftlaquo, 2022 og raquoDansklaquo, 2018. Redaktoslashr af raquoJesus garingr til filmen  Jesusfiguren i moderne filmlaquo, 2007. Medredaktoslashr af raquoKonstellationer  kirkerne og det europaeligiske projektlaquo, 2007, og raquoLivet efter doslashden  i de store verdensreligionerlaquo, 2006. Har skrevet adskillige teologiske artikler om bl.a. forholdet mellem stat og kirke, religion, politik og ret samt Giorgio Agambens politiske teologi.

\end{cvletter}
\end{document}