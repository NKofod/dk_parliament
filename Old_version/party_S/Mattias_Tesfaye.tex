%!TEX TS-program = xelatex
%!TEX encoding = UTF-8 Unicode
\documentclass[11pt, a4paper]{awesome-cv}
\geometry{left=1.4cm, top=.8cm, right=1.4cm, bottom=1.8cm, footskip=.5cm}
\fontdir[fonts/]
\colorlet{awesome}{S-colour}
\setbool{acvSectionColorHighlight}{true}
\renewcommand{\acvHeaderSocialSep}{\quad\textbar\quad}
\recipient{}{}
\name{Mattias}{Tesfaye}
\mobile{+45 7226 8400}
\email{jmjm.dk}
\position{Justitsminister{\enskip\cdotp\enskip}Socialdemokratiet}
\address{}
\photo[circle,noedge,left]{"./party_S/Mattias_Tesfaye_profile.jpg"}
\letterdate{\today}
\lettertitle{Mattias Tesfaye - Blå Bog}
\letteropening{}
\letterclosing{}
\letterenclosure[Attached]{Stemme Statistik}
\begin{document}
\makecvheader[R]
\makecvfooter{\today}{\lettertitle{Mattias Tesfaye - Blå Bog}}{}
\makelettertitle
\begin{cvletter}
\lettersection{Baggrund}
Mattias Tesfaye, foslashdt 31. marts 1981 i Aringrhus, soslashn af Tesfaye Mamo og social og sundhedsassistent Jytte Svensson.nbspGift og har tre boslashrn.

\lettersection{Uddannelse}
\begin{itemize}
\item Uddannet murersvend,19982001.
\end{itemize}
\lettersection{Parlamentarisk Karriere}
\subsection*{Ministerposter}
\begin{itemize}
\item Justitsministerfra 2. maj 2022.
\item Udlændinge og integrationsminister27. juni 2019  2. maj 2022.
\end{itemize}
\subsection*{Folketinget}
\subsubsection*{Medlemsperioder}
\begin{itemize}
\item Folketingsmedlem for Socialdemokratiet i Københavns Omegns Storkreds fra 18. juni 2015.
\end{itemize}
\subsubsection*{Kandidaturer}
\begin{itemize}
\item Kandidat for Socialdemokratiet i Brøndbykredsenfra 2014.
\end{itemize}
\lettersection{Publikationer}
Har skrevet Velkommen Mustafa  50 års socialdemokratisk udlændingepolitik, 2017, Kloge hænder  et forsvar for håndværk og faglighed, 2013, Vi er ikke dyr, men vi er tyskere  Working poor på Danmarks dørtrin, 2010, og Livremmen, 2004.

\end{cvletter}
\end{document}