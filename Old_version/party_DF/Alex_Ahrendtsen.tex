%!TEX TS-program = xelatex
%!TEX encoding = UTF-8 Unicode
\documentclass[11pt, a4paper]{awesome-cv}
\geometry{left=1.4cm, top=.8cm, right=1.4cm, bottom=1.8cm, footskip=.5cm}
\fontdir[fonts/]
\colorlet{awesome}{DF-colour}
\setbool{acvSectionColorHighlight}{true}
\renewcommand{\acvHeaderSocialSep}{\quad\textbar\quad}
\recipient{}{}
\name{Alex}{Ahrendtsen}
\mobile{+45 3337 5105}
\email{alex.ahrendtsenft.dk}
\position{Medlem af Folketinget{\enskip\cdotp\enskip}Dansk Folkeparti}
\address{}
\photo[circle,noedge,left]{"./party_DF/Alex_Ahrendtsen_profile.jpg"}
\letterdate{\today}
\lettertitle{Alex Ahrendtsen - Blå Bog}
\letteropening{}
\letterclosing{}
\letterenclosure[Attached]{Stemme Statistik}
\begin{document}
\makecvheader[R]
\makecvfooter{\today}{\lettertitle{Alex Ahrendtsen - Blå Bog}}{}
\makelettertitle
\begin{cvletter}
\lettersection{Baggrund}
Alex Ahrendtsen, født 14. februar 1967 i Kolding, søn af pensioneret entreprenørformand og tømrer Henning Ahrendtsen og pensioneret korrespondent Marianne Ahrendtsen.

\lettersection{Uddannelse}
\begin{itemize}
\item Cand.mag. i dansk, litteratur, religion og oldgræsk, Odense Universitet,19891996.
\end{itemize}
\lettersection{Parlamentarisk Karriere}
\subsection*{Ordførerskaber}
\begin{itemize}
\item Bygningsordfører
\item Europarådsordfører
\item Børneordfører
\item Boligordfører
\item Erhvervsuddannelsesordfører
\item EUordfører
\item Færdselsordfører
\item Folkeskoleordfører
\item Gymnasieordfører
\item Transportordfører
\item Undervisningsordfører
\end{itemize}
\subsection*{Parlamentariske Tillidsposter}
\begin{itemize}
\item Børneordførerfra 2022.
\item Erhvervsuddannelsesordførerfra 2022.
\item Europarådsordførerfra 2022.
\item Færdselsordførerfra 2022.
\item Gymnasieordførerfra 2022.
\item Transportordførerfra 2022.
\item Undervisningsordførerfra 2022.
\item EUordførerfra 2021.
\item Bolig og bygningsordførerfra 2020.
\item Ordfører for udviklingsbistandfra 2020.
\item Idrætsordfører20192020.
\item Medieordfører20192020.
\item Folkeskoleordførerfra 2011.
\item Kulturordfører20112020.
\end{itemize}
\subsection*{Folketinget}
\subsubsection*{Medlemsperioder}
\begin{itemize}
\item Folketingsmedlem for Dansk Folkeparti i Fyns Storkreds fra 15. september 2011.
\end{itemize}
\subsubsection*{Kandidaturer}
\begin{itemize}
\item Kandidat for Dansk Folkeparti i Odense Sydkredsenfra 2007.
\item Kandidat for Dansk Folkeparti i Svendborgkredsen20012007.
\end{itemize}
\lettersection{Erhvervserfaring}
\begin{itemize}
\item Forlægger, Trykkefrihedsselskabets Bibliotek, Odense,20072014.
\item Forlægger, Lysias,20032020.
\item Forfatter på forskellige forlag,fra 1997.
\item Receptionist, underviser, redaktør, tekstforfatter, produktionsmedhjælper og selvstændig,19962007.
\end{itemize}
\lettersection{Publikationer}
Har skrevet Når danskere bøjer af  Islamiseringen af Odense, Trykkefrihedsselskabets Bibliotek, 2009, Danmark  Fortællinger ved kornmod, Lysias, 2003, Den danske ligevægt, Frihedsbrevet, 2002, Martin A. Hansen og Indre Mission, Odense Universitetsforlag, 1997 og Indsigt eller fordom  missionen i litteraturen fra Aakjær til Høeg, Lohse, 1997.

\end{cvletter}
\end{document}