%!TEX TS-program = xelatex
%!TEX encoding = UTF-8 Unicode
\documentclass[11pt, a4paper]{awesome-cv}
\geometry{left=1.4cm, top=.8cm, right=1.4cm, bottom=1.8cm, footskip=.5cm}
\fontdir[fonts/]
\colorlet{awesome}{EL-colour}
\setbool{acvSectionColorHighlight}{true}
\renewcommand{\acvHeaderSocialSep}{\quad\textbar\quad}
\recipient{}{}
\name{Henning}{Hyllested}
\mobile{+45 3337 5010}
\email{henning.hyllested@ft.dk}
\position{Medlem af Folketinget{\enskip\cdotp\enskip}Enhedslisten}
\address{}
\photo[circle,noedge,left]{"./party_Enhedslisten/Henning_Hyllested_profile.jpg"}
\letterdate{\today}
\lettertitle{Henning Hyllested - Blå Bog}
\letteropening{}
\letterclosing{}
\letterenclosure[Attached]{Stemme Statistik}
\begin{document}
\makecvheader[R]
\makecvfooter{\today}{\lettertitle{Henning Hyllested - Blå Bog}}{}
\makelettertitle
\begin{cvletter}
\lettersection{Baggrund}
Henning Hyllested, født 28. februar 1954 i Esbjerg, søn af Ronald Søren Hyllested og Annegrethe Hyllested. Gift med beskæftigelsesvejleder Ina Birgitte Thygesen.

\lettersection{Parlamentarisk Karriere}
\subsection*{Ordførerskaber}
\begin{itemize}
\item Landdistrikts- og øordfører
\item Transportordfører
\item Turismeordfører
\end{itemize}
\subsection*{Parlamentariske Tillidsposter}
\begin{itemize}
\item Landdistrikts- og øordfører, turismeordfører fra 2015.
\item Transportordfører fra 2011.
\end{itemize}
\subsection*{Folketinget}
\subsubsection*{Medlemsperioder}
\begin{itemize}
\item Folketingsmedlem for Enhedslisten i Sydjyllands Storkreds fra 15. september 2011.
\end{itemize}
\subsubsection*{Kandidaturer}
\begin{itemize}
\item Kandidat for Enhedslisten i Esbjerg Bykredsen fra 2007.
\item Kandidat for Enhedslisten i Esbjergkredsen 1998-2006.
\end{itemize}
\lettersection{Erhvervserfaring}
\begin{itemize}
\item Havnearbejder, Esbjerg Havn, 1981-2011.
\item Fabriksarbejder, Esbjerg, 1980-1980.
\item Rejsesekretær, Danmarks Kommunistiske Ungdomsforbund, 1978-1980.
\item Fabriksarbejder, Esbjerg, 1973-1978.
\end{itemize}
\end{cvletter}
\end{document}