%!TEX TS-program = xelatex
%!TEX encoding = UTF-8 Unicode
\documentclass[11pt, a4paper]{awesome-cv}
\geometry{left=1.4cm, top=.8cm, right=1.4cm, bottom=1.8cm, footskip=.5cm}
\fontdir[fonts/]
\colorlet{awesome}{EL-colour}
\setbool{acvSectionColorHighlight}{true}
\renewcommand{\acvHeaderSocialSep}{\quad\textbar\quad}
\recipient{}{}
\name{Rune}{Lund}
\mobile{+45 3337 5016}
\email{rune.lund@ft.dk}
\position{Medlem af Folketinget{\enskip\cdotp\enskip}Enhedslisten}
\address{}
\photo[circle,noedge,left]{"./party_Enhedslisten/Rune_Lund_profile.jpg"}
\letterdate{\today}
\lettertitle{Rune Lund - Blå Bog}
\letteropening{}
\letterclosing{}
\letterenclosure[Attached]{Stemme Statistik}
\begin{document}
\makecvheader[R]
\makecvfooter{\today}{\lettertitle{Rune Lund - Blå Bog}}{}
\makelettertitle
\begin{cvletter}
\lettersection{Baggrund}
Rune Lund, født 30. oktober 1976 i København, søn af journalist Kurt Lund og lektor Gunhild Paaske. Gift. Har tre børn.

\lettersection{Uddannelse}
\begin{itemize}
\item Statskundskab, cand.scient.pol., Københavns Universitet, 1998-2007.
\item Student, Københavns åbne Gymnasium (tidligere Vestre Borgerdyd Gymnasium), 1992-1995.
\item Folkeskole, Valby Skole og Mariendal Friskole, 1983-1992.
\end{itemize}
\lettersection{Parlamentarisk Karriere}
\subsection*{Ordførerskaber}
\begin{itemize}
\item Finansordfører
\item Skatteordfører
\end{itemize}
\subsection*{Parlamentariske Tillidsposter}
\begin{itemize}
\item Næstformand for Enhedslistens folketingsgruppe fra 2020.
\item Gruppesekretær for Enhedslistens folketingsgruppe fra 2019.
\item Finansordfører fra 2018.
\item Grundlovsordfører 2017-2018.
\item Retsordfører 2016-2018.
\item Kommunalordfører og it-ordfører 2015-2016.
\item Skatteordfører fra 2015.
\item Formand for Enhedslistens folketingsgruppe 2006-2007.
\item Udenrigsordfører, forsvarsordfører, EU-ordfører og trafikordfører 2005-2007.
\end{itemize}
\subsection*{Folketinget}
\subsubsection*{Medlemsperioder}
\begin{itemize}
\item Folketingsmedlem for Enhedslisten i Københavns Storkreds fra 5. juni 2019.
\item Folketingsmedlem for Enhedslisten i Fyns Storkreds 18. juni 2015 - 5. juni 2019.
\item Folketingsmedlem for Enhedslisten i Fyns Amtskreds 8. februar 2005 - 13. november 2007.
\end{itemize}
\subsubsection*{Kandidaturer}
\begin{itemize}
\item Kandidat for Enhedslisten i Middelfartkredsen fra 2014.
\item Kandidat for Enhedslisten i Odense Østkredsen 2003-2008.
\item Kandidat for Enhedslisten i Otterupkredsen 2003-2008.
\end{itemize}
\lettersection{Erhvervserfaring}
\begin{itemize}
\item Økonomisk konsulent, Gladsaxe Kommune, Social- og Sundhedsforvaltningen, 2013-2015.
\item Økonomisk konsulent, New York, USA, 2011-2013.
\item Økonomisk konsulent, Gladsaxe Kommune, Social- og Sundhedsforvaltningen, 2009-2011.
\item Selvstændig, iværksætter, København, 2007-2008.
\item Sekretariatsleder og koordinator, NGO Forum - Stop Volden, 2002-2003.
\item Pædagogmedhjælper, Vangedehuse, 1998-2001.
\item Hjemmehjælper, Holte, 1996-1997.
\end{itemize}
\end{cvletter}
\end{document}