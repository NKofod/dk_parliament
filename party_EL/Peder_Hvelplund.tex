%!TEX TS-program = xelatex
%!TEX encoding = UTF-8 Unicode
\documentclass[11pt, a4paper]{awesome-cv}
\geometry{left=1.4cm, top=.8cm, right=1.4cm, bottom=1.8cm, footskip=.5cm}
\fontdir[fonts/]
\colorlet{awesome}{EL-colour}
\setbool{acvSectionColorHighlight}{true}
\renewcommand{\acvHeaderSocialSep}{\quad\textbar\quad}
\recipient{}{}
\name{Peder}{Hvelplund}
\mobile{+45 61623231}
\email{peder.hvelplund@ft.dk}
\position{Medlem af Folketinget{\enskip\cdotp\enskip}Enhedslisten}
\address{}
\photo[circle,noedge,left]{"./party_Enhedslisten/Peder_Hvelplund_profile.jpg"}
\letterdate{\today}
\lettertitle{Peder Hvelplund - Blå Bog}
\letteropening{}
\letterclosing{}
\letterenclosure[Attached]{Stemme Statistik}
\begin{document}
\makecvheader[R]
\makecvfooter{\today}{\lettertitle{Peder Hvelplund - Blå Bog}}{}
\makelettertitle
\begin{cvletter}
\lettersection{Baggrund}
Peder Hvelplund, født 8. september 1967 i Ringkøbing, søn af brugsuddeler Svend Aksel Hvelplund og hjemmegående Kirstine Hvelplund. Gift med Louise Bilde Hvelplund. Har børnene Oskar, Anna og Karl Bilde Hvelplund. 

\lettersection{Uddannelse}
\begin{itemize}
\item Fritidspædagog, Hjørring Seminarium, 1990-1993.
\item Gymnasium, Ringkøbing Gymnasium, 1983-1986.
\item Folkeskole, Tim Skole, 1981-1983.
\item Folkeskole, Hover-Thorsted Friskole, 1974-1981.
\end{itemize}
\lettersection{Parlamentarisk Karriere}
\subsection*{Ordførerskaber}
\begin{itemize}
\item Indfødsretsordfører
\item Klimaordfører
\item Naturordfører
\end{itemize}
\subsection*{Parlamentariske Tillidsposter}
\begin{itemize}
\item Coronaordfører fra 2021.
\item Natur- og klimaordfører fra 2021.
\item Formand for Enhedslistens folketingsgruppe fra 2020.
\item Medlem af Enhedslistens gruppeledelse fra 2019.
\item Næstformand for Enhedslistens folketingsgruppe fra 2019.
\item Indfødsretsordfører fra 2019.
\item Ældreordfører 2018-2021.
\item Sundheds- og psykiatriordfører 2016-2021.
\end{itemize}
\subsection*{Folketinget}
\subsubsection*{Medlemsperioder}
\begin{itemize}
\item Folketingsmedlem for Enhedslisten i Nordjyllands Storkreds fra 5. juni 2019.
\end{itemize}
\subsubsection*{Kandidaturer}
\begin{itemize}
\item Kandidat for Enhedslisten i Brønderslevkredsen fra 2015.
\item Kandidat for Enhedslisten i Hjørringkredsen fra 2007.
\end{itemize}
\lettersection{Erhvervserfaring}
\begin{itemize}
\item Opsøgende medarbejder i udsatteområdet, Hjørring Kommune, 2013-2019.
\item Social mentor i beskæftigelsesområdet, Hjørring Kommune, 1994-2013.
\end{itemize}
\end{cvletter}
\end{document}