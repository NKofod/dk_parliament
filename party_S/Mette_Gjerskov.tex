%!TEX TS-program = xelatex
%!TEX encoding = UTF-8 Unicode
\documentclass[11pt, a4paper]{awesome-cv}
\geometry{left=1.4cm, top=.8cm, right=1.4cm, bottom=1.8cm, footskip=.5cm}
\fontdir[fonts/]
\colorlet{awesome}{S-colour}
\setbool{acvSectionColorHighlight}{true}
\renewcommand{\acvHeaderSocialSep}{\quad\textbar\quad}
\recipient{}{}
\name{Mette}{Gjerskov}
\mobile{+45 3337 4019}
\email{mette.gjerskov@ft.dk}
\position{Fhv. minister, Cand.agro.{\enskip\cdotp\enskip}Socialdemokratiet}
\address{}
\photo[circle,noedge,left]{"./party_S/Mette_Gjerskov_profile.jpg"}
\letterdate{\today}
\lettertitle{Mette Gjerskov - Blå Bog}
\letteropening{}
\letterclosing{}
\letterenclosure[Attached]{Stemme Statistik}
\begin{document}
\makecvheader[R]
\makecvfooter{\today}{\lettertitle{Mette Gjerskov - Blå Bog}}{}
\makelettertitle
\begin{cvletter}
\lettersection{Baggrund}
Mette Gjerskov, født 28. juli 1966 i Gundsø, datter af fhv. skyldrådsformand Gert Gjerskov og pensioneret overlærer Louise Gjerskov.

\lettersection{Uddannelse}
\begin{itemize}
\item Cand.agro., Den Kongelige Veterinær- og Landbohøjskole, 1987-1993.
\item Matematik, fysik og kemi, VUC Ballerup, 1986-1987.
\item Samfundssproglig student, Amtsgymnasiet i Roskilde, 1982-1985.
\end{itemize}
\lettersection{Parlamentarisk Karriere}
\subsection*{Ministerposter}
\begin{itemize}
\item Minister for fødevarer, landbrug og fiskeri 3. oktober 2011 - 9. august 2013.
\end{itemize}
\subsection*{Ordførerskaber}
\begin{itemize}
\item Miljøordfører
\item Verdensmålsordfører
\end{itemize}
\subsection*{Parlamentariske Tillidsposter}
\begin{itemize}
\item Næstformand i Finansudvalgets arbejdsgruppe om FN's Verdensmål fra 2019.
\item Miljøordfører fra 2019.
\item Næstformand for Folketingets 2030 netværk om FN's Verdensmål fra 2019.
\item Verdensmålsordfører fra 2019.
\item Formand for Europaudvalget 2015-2016.
\item Medlem af Folketingets Tværpolitiske Netværk for Seksuel og Reproduktiv Sundhed og Rettigheder (formand 2014-2019, næstformand fra 2019) fra 2014.
\item Formand for Det Udenrigspolitiske Nævn 2013-2015.
\end{itemize}
\subsection*{Folketinget}
\subsubsection*{Medlemsperioder}
\begin{itemize}
\item Folketingsmedlem for Socialdemokratiet i Sjællands Storkreds fra 13. november 2007.
\item Folketingsmedlem for Socialdemokratiet i Roskilde Amtskreds 8. februar 2005 - 13. november 2007.
\end{itemize}
\subsubsection*{Kandidaturer}
\begin{itemize}
\item Kandidat for Socialdemokratiet i Roskildekredsen fra 2002.
\end{itemize}
\lettersection{Erhvervserfaring}
\begin{itemize}
\item Centerleder, Center for frivilligt socialt arbejde, 2004-2005.
\item Fuldmægtig, BioTIK-sekretariatet, Forbrugerstyrelsen, 2001-2004.
\item Afdelingsleder, Direktoratet for FødevareErhverv, 1999-2001.
\item Fuldmægtig, Fødevareministeriet, 1995-1999.
\item Forskellige ansættelser som lærervikar, pædagogmedhjælper, receptionist og landbrugsmedhjælper, 1980-1995.
\end{itemize}
\end{cvletter}
\end{document}