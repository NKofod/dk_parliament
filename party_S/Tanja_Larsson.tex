%!TEX TS-program = xelatex
%!TEX encoding = UTF-8 Unicode
\documentclass[11pt, a4paper]{awesome-cv}
\geometry{left=1.4cm, top=.8cm, right=1.4cm, bottom=1.8cm, footskip=.5cm}
\fontdir[fonts/]
\colorlet{awesome}{S-colour}
\setbool{acvSectionColorHighlight}{true}
\renewcommand{\acvHeaderSocialSep}{\quad\textbar\quad}
\recipient{}{}
\name{Tanja}{Larsson}
\mobile{+45 3337 4089}
\email{tanja.larsson@ft.dk}
\position{Medlem af Folketinget{\enskip\cdotp\enskip}Socialdemokratiet}
\address{}
\photo[circle,noedge,left]{"./Tanja_Larsson_profile.jpg"}
\letterdate{\today}
\lettertitle{Tanja Larsson - Blå Bog}
\letteropening{}
\letterclosing{}
\letterenclosure[Attached]{Stemme Statistik}
\begin{document}
\makecvheader[R]
\makecvfooter{\today}{\lettertitle{Tanja Larsson - Blå Bog}}{}
\makelettertitle
\begin{cvletter}
\lettersection{Baggrund}
Tanja Larsson, født 7. juli 1977 i Haslev, datter af daginstitutionsleder Thomas Larsson og børnehaveklasselærer Lis Pedersen.

\lettersection{Parlamentarisk Karriere}
\subsection*{Ordførerskaber}
\begin{itemize}
\item Friskole- og efterskoleordfører
\item Boligordfører
\item Færøerneordfører
\end{itemize}
\subsection*{Parlamentariske Tillidsposter}
\begin{itemize}
\item Boligordfører fra 2020.
\item Færdselssikkerhedsordfører 2019-2020.
\item LGBT-ordfører 2019-2020.
\end{itemize}
\subsection*{Folketinget}
\subsubsection*{Medlemsperioder}
\begin{itemize}
\item Folketingsmedlem for Socialdemokratiet i Sjællands Storkreds fra 1. oktober 2019.
\end{itemize}
\subsubsection*{Kandidaturer}
\begin{itemize}
\item Kandidat for Socialdemokratiet i Faxekredsen fra 2018.
\end{itemize}
\end{cvletter}
\end{document}