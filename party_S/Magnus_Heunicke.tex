%!TEX TS-program = xelatex
%!TEX encoding = UTF-8 Unicode
\documentclass[11pt, a4paper]{awesome-cv}
\geometry{left=1.4cm, top=.8cm, right=1.4cm, bottom=1.8cm, footskip=.5cm}
\fontdir[fonts/]
\colorlet{awesome}{S-colour}
\setbool{acvSectionColorHighlight}{true}
\renewcommand{\acvHeaderSocialSep}{\quad\textbar\quad}
\recipient{}{}
\name{Magnus}{Heunicke}
\mobile{+45 7226 9000}
\email{sum@sum.dk}
\position{Sundhedsminister{\enskip\cdotp\enskip}Socialdemokratiet}
\address{}
\photo[circle,noedge,left]{"./Magnus_Heunicke_profile.jpg"}
\letterdate{\today}
\lettertitle{Magnus Heunicke - Blå Bog}
\letteropening{}
\letterclosing{}
\letterenclosure[Attached]{Stemme Statistik}
\begin{document}
\makecvheader[R]
\makecvfooter{\today}{\lettertitle{Magnus Heunicke - Blå Bog}}{}
\makelettertitle
\begin{cvletter}
\lettersection{Baggrund}
Magnus Johannes Heunicke, født 28. januar 1975 i Næstved, søn af borgmester Henning Jensen og specialklasselærer Inger Heunicke. Gift med direktør Nina Groes. Far til Ella Heunicke Groes og Rakel Heunicke Groes.

\lettersection{Uddannelse}
\begin{itemize}
\item Journalist, Danmarks Journalisthøjskole, Aarhus, 1998-2002.
\item Student, Næstved Gymnasium, 1992-1995.
\item 10. klasses afgangsprøve, Gunslevholm Idrætsefterskole, Falster, 1991-1992.
\item Folkeskolens 9. klasses afgangsprøve, Kildemarkskolen, Næstved, 1981-1991.
\end{itemize}
\lettersection{Parlamentarisk Karriere}
\subsection*{Parlamentariske Tillidsposter}
\begin{itemize}
\item Næstformand for Udvalget for Videnskab og Teknologi 2005-2007.
\item Tidligere politisk ordfører, trafikordfører, it- og teleordfører, velfærdsordfører, kommunalordfører og ordfører vedrørende landdistrikter og øer.
\end{itemize}
\subsection*{Folketinget}
\subsubsection*{Medlemsperioder}
\begin{itemize}
\item Folketingsmedlem for Socialdemokratiet i Sjællands Storkreds fra 13. november 2007.
\item Folketingsmedlem for Socialdemokratiet i Storstrøms Amtskreds 8. februar 2005 - 13. november 2007.
\end{itemize}
\subsubsection*{Kandidaturer}
\begin{itemize}
\item Kandidat for Socialdemokratiet i Næstvedkredsen fra 2004.
\end{itemize}
\lettersection{Erhvervserfaring}
\begin{itemize}
\item Journalist, Danmarks Radio, 2001-2005.
\item Rejsesekretær, Danmarks Socialdemokratiske Ungdom, 1996-1998.
\item Piccolo og projektmedarbejder, Kommunernes Landsforening, 1995-1996.
\item Scenemester, Bio Næstved, 1992-1995.
\item Opvasker, Bilkas Bistro, Næstved Storcenter, 1990-1991.
\end{itemize}
\end{cvletter}
\end{document}