%!TEX TS-program = xelatex
%!TEX encoding = UTF-8 Unicode
\documentclass[11pt, a4paper]{awesome-cv}
\geometry{left=1.4cm, top=.8cm, right=1.4cm, bottom=1.8cm, footskip=.5cm}
\fontdir[fonts/]
\colorlet{awesome}{S-colour}
\setbool{acvSectionColorHighlight}{true}
\renewcommand{\acvHeaderSocialSep}{\quad\textbar\quad}
\recipient{}{}
\name{Nick}{Hækkerup}
\mobile{+45 7226 8400}
\email{jm@jm.dk}
\position{Justitsminister{\enskip\cdotp\enskip}Socialdemokratiet}
\address{}
\photo[circle,noedge,left]{"./party_S/Nick_Hækkerup_profile.jpg"}
\letterdate{\today}
\lettertitle{Nick Hækkerup - Blå Bog}
\letteropening{}
\letterclosing{}
\letterenclosure[Attached]{Stemme Statistik}
\begin{document}
\makecvheader[R]
\makecvfooter{\today}{\lettertitle{Nick Hækkerup - Blå Bog}}{}
\makelettertitle
\begin{cvletter}
\lettersection{Baggrund}
Nick Hækkerup, født 3. april 1968, søn af fhv. borgmester og fhv. MF Klaus Hækkerup og fhv. skoleinspektør Irene Hækkerup. Gift med landskabsarkitekt Petra Freisleben Hækkerup. Har børnene Fie, født i 1994, Mille, født i 1998, Emil, født i 2003 og Malthe, født i 2006.

\lettersection{Uddannelse}
\begin{itemize}
\item Ph.d., Københavns Universitet, 1994-1998.
\item Cand.jur., Københavns Universitet, 1988-1994.
\end{itemize}
\lettersection{Parlamentarisk Karriere}
\subsection*{Ministerposter}
\begin{itemize}
\item Justitsminister fra 27. juni 2019.
\item Minister for sundhed og forebyggelse 3. februar 2014 - 28. juni 2015.
\item Handels- og europaminister 9. august 2013 - 3. februar 2014.
\item Forsvarsminister 3. oktober 2011 - 9. august 2013.
\end{itemize}
\subsection*{Parlamentariske Tillidsposter}
\begin{itemize}
\item Næstformand for Det Udenrigspolitiske Nævn 2015-2019.
\item Medlem af Udenrigsudvalget (formand 2015-2016) 2015-2019.
\item Medlem af Finansudvalget og Skatteudvalget 2007-2011.
\item Næstformand for Socialdemokratiet 2005-2012.
\end{itemize}
\subsection*{Folketinget}
\subsubsection*{Medlemsperioder}
\begin{itemize}
\item Folketingsmedlem for Socialdemokratiet i Nordsjællands Storkreds fra 13. november 2007.
\end{itemize}
\subsubsection*{Kandidaturer}
\begin{itemize}
\item Kandidat for Socialdemokratiet i Hillerødkredsen fra 2018.
\item Kandidat for Socialdemokratiet i Egedalkredsen 2007-2018.
\end{itemize}
\lettersection{Erhvervserfaring}
\begin{itemize}
\item Underviser i forfatningsret, Københavns Universitet, 2017-2019.
\item Adjunkt, Københavns Universitet, 1998-2000.
\item Fuldmægtig ved Landsskatteretten, 1994-1994.
\end{itemize}
\lettersection{Publikationer}
Har skrevet ph.d.-afhandlingen »Kontrol og sanktioner i EF-retten«, 1998. Medforfatter til »Sandheden Kort - Christiansborg fra A til Å«, People's Press, 2018, »Controls and Sanctions in the EU Law«, Djøf Forlag, 2001 og »Udvikling i EU siden 1992 på de områder, der er omfattet af de danske forbehold« Dansk Udenrigspolitisk Institut (DUPI), 2001.

\end{cvletter}
\end{document}