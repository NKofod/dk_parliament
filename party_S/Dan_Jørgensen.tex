%!TEX TS-program = xelatex
%!TEX encoding = UTF-8 Unicode
\documentclass[11pt, a4paper]{awesome-cv}
\geometry{left=1.4cm, top=.8cm, right=1.4cm, bottom=1.8cm, footskip=.5cm}
\fontdir[fonts/]
\colorlet{awesome}{S-colour}
\setbool{acvSectionColorHighlight}{true}
\renewcommand{\acvHeaderSocialSep}{\quad\textbar\quad}
\recipient{}{}
\name{Dan}{Jørgensen}
\mobile{+45 3392 2800}
\email{kefm@kefm.dk}
\position{Klima-, energi- og forsyningsminister{\enskip\cdotp\enskip}Socialdemokratiet}
\address{}
\photo[circle,noedge,left]{"./party_S/Dan_Jørgensen_profile.jpg"}
\letterdate{\today}
\lettertitle{Dan Jørgensen - Blå Bog}
\letteropening{}
\letterclosing{}
\letterenclosure[Attached]{Stemme Statistik}
\begin{document}
\makecvheader[R]
\makecvfooter{\today}{\lettertitle{Dan Jørgensen - Blå Bog}}{}
\makelettertitle
\begin{cvletter}
\lettersection{Baggrund}
Dan Jannik Jørgensen, født 12. juni 1975 i Odense.



\lettersection{Uddannelse}
\begin{itemize}
\item Cand. scient. pol., Aarhus Universitet, 2004-2004.
\end{itemize}
\lettersection{Parlamentarisk Karriere}
\subsection*{Ministerposter}
\begin{itemize}
\item Klima-, energi- og forsyningsminister fra 27. juni 2019.
\item Minister for fødevarer, landbrug og fiskeri 12. december 2013 - 28. juni 2015.
\end{itemize}
\subsection*{Parlamentariske Tillidsposter}
\begin{itemize}
\item Næstformand for Socialdemokratiets folketingsgruppe 2017-2019.
\item Næstformand for den danske delegation til NATO's Parlamentariske Forsamling 2015-2019.
\item Medlem af Europa-Parlamentet (formand for Animal Welfare Intergroup, næstformand for Europa-Parlamentets udvalg for miljø, folkesundhed og fødevaresikkerhed, formand for de danske socialdemokrater i Europa-Parlamentet) 2004-2013.
\end{itemize}
\subsection*{Folketinget}
\subsubsection*{Medlemsperioder}
\begin{itemize}
\item Folketingsmedlem for Socialdemokratiet i Fyns Storkreds fra 18. juni 2015.
\end{itemize}
\subsubsection*{Kandidaturer}
\begin{itemize}
\item Kandidat for Socialdemokratiet i Middelfartkredsen fra 2011.
\end{itemize}
\lettersection{Erhvervserfaring}
\begin{itemize}
\item Adjungeret professor, Aalborg Universitet, 2016-2019.
\item Ekstern lektor, Institut for Statskundskab, Københavns Universitet, 2013-2013.
\item Ekstern lektor, Seattle University, 2012-2013.
\item Ekstern lektor, Sciences Po, Paris, 2012-2013.
\item Ekstern lektor, DIS, Danish Institute for Study Abroad, København, 2011-2013.
\item Ekstern lektor, Institut for Statskundskab, Aarhus Universitet, 2010-2010.
\end{itemize}
\lettersection{Publikationer}
Har skrevet »Staunings arv - vejen til et lykkeligt Danmark«, People's Press, 2018. »Grønt håb - Klimapolitik 2.0«, Forlaget Sohn, 2010. »Mellem Mars og Venus - EU's rolle i fremtidens verdensorden«, Forlaget Sohn, 2009. »Politikere med begge ben på jorden hænger ikke på træerne«, Informations Forlag, 2009. »Grøn Globalisering - miljøpolitik i forandring«, Hovedland, 2007 og »Eurovisioner - Essays om fremtidens Europa«, Informations Forlag, 2006. Medforfatter til »Beyond deniers and belivers - towards a map of the politics of climate change« i tidsskriftet Global Environmental Change, 2015. 

\end{cvletter}
\end{document}