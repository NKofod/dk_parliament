%!TEX TS-program = xelatex
%!TEX encoding = UTF-8 Unicode
\documentclass[11pt, a4paper]{awesome-cv}
\geometry{left=1.4cm, top=.8cm, right=1.4cm, bottom=1.8cm, footskip=.5cm}
\fontdir[fonts/]
\colorlet{awesome}{S-colour}
\setbool{acvSectionColorHighlight}{true}
\renewcommand{\acvHeaderSocialSep}{\quad\textbar\quad}
\recipient{}{}
\name{Mogens}{Jensen}
\mobile{+45 3337 5500}
\email{mogens.jensen@ft.dk}
\position{Fhv. minister{\enskip\cdotp\enskip}Socialdemokratiet}
\address{}
\photo[circle,noedge,left]{"./party_S/Mogens_Jensen_profile.jpg"}
\letterdate{\today}
\lettertitle{Mogens Jensen - Blå Bog}
\letteropening{}
\letterclosing{}
\letterenclosure[Attached]{Stemme Statistik}
\begin{document}
\makecvheader[R]
\makecvfooter{\today}{\lettertitle{Mogens Jensen - Blå Bog}}{}
\makelettertitle
\begin{cvletter}
\lettersection{Baggrund}
Mogens Jensen, født 31. oktober 1963 i Nykøbing Mors, søn af ekspedient Harry Jensen og hjemmehjælper Ebba Møller Jensen.

\lettersection{Uddannelse}
\begin{itemize}
\item Fagbevægelsens lederuddannelse, LO-Skolen, Helsingør, 1997-1999.
\item Student, Morsø Gymnasium, 1979-1982.
\item Nykøbing Mors Folke- og Realskole, 1970-1979.
\end{itemize}
\lettersection{Parlamentarisk Karriere}
\subsection*{Ministerposter}
\begin{itemize}
\item Minister for fødevarer, fiskeri og ligestilling og minister for nordisk samarbejde 27. juni 2019 - 19. november 2020.
\item Handels- og udviklingsminister 3. februar 2014 - 28. juni 2015.
\end{itemize}
\subsection*{Parlamentariske Tillidsposter}
\begin{itemize}
\item Næstformand for Kulturudvalget fra 2020.
\item Næstformand for Retsudvalget 2015-2019.
\item Næstformand for Socialdemokratiet fra 2012.
\item Formand for den socialdemokratiske folketingsgruppe 2011-2012.
\item Tidligere formand og næstformand for delegationen til Europarådets parlamentariske forsamling (PACE).
\item Tidligere ordfører for Grønland, Færøerne og Nordisk Råd.
\item Tidligere ordfører for kultur, medier og idræt.
\end{itemize}
\subsection*{Folketinget}
\subsubsection*{Medlemsperioder}
\begin{itemize}
\item Folketingsmedlem for Socialdemokratiet i Vestjyllands Storkreds fra 13. november 2007.
\item Folketingsmedlem for Socialdemokratiet i Ringkøbing Amtskreds 8. februar 2005 - 13. november 2007.
\end{itemize}
\subsubsection*{Kandidaturer}
\begin{itemize}
\item Kandidat for Socialdemokratiet i Herning Sydkredsen fra 2007.
\item Kandidat for Socialdemokratiet i Herningkredsen 2003-2006.
\end{itemize}
\lettersection{Erhvervserfaring}
\begin{itemize}
\item Konsulent, LO, 1987-2005.
\item Ulandskonsulent, Arbejdernes Oplysningsforbund, 1986-1987.
\item Kulturkonsulent, Arbejdernes Oplysningsforbund, 1985-1986.
\item Uddannelsessekretær, Danmarks Socialdemokratiske Ungdom, 1982-1985.
\end{itemize}
\lettersection{Publikationer}
Har bl.a. skrevet debatoplæggene »En samordnet ungdomsuddannelse«, 1985, og »Menneskene skal blomstre«, 1987. Bidrag til LO's 100-årsjubilæumsbog »I takt med tiden«, 1998. Diverse film- og teaterproduktioner.

\end{cvletter}
\end{document}